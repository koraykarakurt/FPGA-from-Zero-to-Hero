\documentclass{article}
\usepackage{graphicx} % For including diagrams
\usepackage{amsmath}  % For mathematical equations
\usepackage{amssymb}  % For additional math symbols
\usepackage{hyperref} % For hyperlinks
\usepackage{listings} % For code snippets

\title{Understanding Metastability in FPGAs}
\author{Mert Ecevit}
\date{\today}

\begin{document}

\maketitle
\newpage % Forces a page break to separate the title page

\begin{abstract}
A condition known as metastability occurs when a signal is passed across circuits in unrelated or asynchronous clock domains, which can lead to system failure in FPGAs. This document abstracts the \textsc{Altera} white paper named "Understanding Metastability in FPGAs".
\end{abstract}

\section{Introduction}
\begin{itemize}
    \item Importance of timing in digital circuits
    \item Definition of metastability
    \item Why metastability is a concern in FPGAs
\end{itemize}
To ensure reliable operation, a register's input must be constant for a predetermined period of time before and after the clock edge. If a data signal transition does not satisfy the setup-time or hold-time requirements of the register, the output of the register may enter a metastable situation. A steady, defined state is frequently returned rapidly by registers.\cite{altera_metastability}

\begin{figure}[htbp]
    \centering
    \includegraphics[width=1\linewidth]{Figures/Ekran görüntüsü 2025-03-19 175615.png}
    \caption{Visualisation of Metastability}
    \label{fig:enter-label}
\end{figure}

\section{Metastability in FPGA Designs}
\begin{itemize}
    \item Impact on synchronous and asynchronous designs
    \item Cross-domain clock transfer challenges
    \item Examples of metastability issues in FPGA circuits
\end{itemize}
After beginning in a low state, the data output signal becomes metastable and oscillates between high and low states.

\begin{figure}[htbp]
    \centering
    \includegraphics[width=1\linewidth]{Figures/Ekran görüntüsü 2025-03-19 182526.png}
    \caption{Example of Metastable Output Signals}
    \label{fig:enter-label}
\end{figure}

\section{Mitigation Techniques}
\begin{itemize}
    \item Use of multi-stage synchronizers
    \item Handshaking protocols
    \item FIFO-based solutions
    \item CDC (Clock Domain Crossing) design best practices
\end{itemize}
The metastable signal has no adverse effect on system operation if the data output signal resolves to a legitimate state prior to the subsequent register capturing the data. However, the system may fail if the metastable signal does not resolve to a low or high state before it reaches the next design register.

\section{Practical Considerations and Simulations}
\begin{itemize}
    \item Simulation of metastability events
    \item Measuring MTBF (Mean Time Between Failures)
    \item FPGA vendor-specific solutions (Xilinx, Intel, etc.)
\end{itemize}
A signal must be synchronized to a new clock domain before it can be utilized as it moves across circuitry in unrelated or asynchronous clock domains. To overcome failures, a synchronizer chain is employed. The time available for a metastable signal to settle is known as the metastability settling time, and it is found in the timing slack present in the synchronizer register to register routes.

\begin{itemize}
    \item The registers in the chain are all clocked by the same or phase-related clocks.
    \item The first register in the chain is driven from an unrelated clock domain, or asynchronously.
    \item Each register fans out to only one register, except the last register in the chain.
\end{itemize}

\begin{figure}[htbp]
    \centering
    \includegraphics[width=1\linewidth]{Figures/Ekran görüntüsü 2025-03-19 183422.png}
    \caption{Synchronizer Chain}
    \label{fig:enter-label}
\end{figure}

The calculation of the MTBF is:
\begin{figure}[htbp]
    \centering
    \includegraphics[width=0.5\linewidth]{Figures/Ekran görüntüsü 2025-03-19 183702.png}
    \caption{Formula for Calculating MTBF}
    \label{fig:enter-label}
\end{figure}

\begin{itemize}
    \item \( C_1, C_2 \) are constants.
    \item \( f_{\text{clk}} \) is the clock frequency.
    \item \( f_{\text{data}} \) is the toggling frequency of the asynchronous input data signal.
    \item \( t_{\text{met}} \) is the available metastability settling time or timing slack available beyond the register’s \( t_{\text{co}} \).
\end{itemize}

\section{Characterizing Metastability Constants}
FPGA vendors can characterize the FPGA for metastability in order to identify the constant parameters in the MTBF equation. When testing MTBF under unrealistic circumstances, a test circuit may be employed. A test circuit is available for the characterizing constants.

\begin{figure}[htbp]
    \centering
    \includegraphics[width=1\linewidth]{Figures/Ekran görüntüsü 2025-03-19 184432.png}
    \caption{A Test Circuit for Calculating MTBF}
    \label{fig:enter-label}
\end{figure}
There are two separate clocks in this design. Every clock cycle, the synchronizer's data input toggles (\( f_{\text{data}} \)). There are two locations for the synchronizing register. The output of the synchronizer is recorded in destination registers one clock cycle and one half-clock cycle later. The circuit recognizes that the sample signals are distinct and emits an error signal if the signal becomes metastable before resolving at the subsequent clock edge. A significant percentage of the metastability events that take place at the half-clock cycle time are detected by this circuitry. A logarithmic scale is used to illustrate the MTBF versus \( t_{\text{met}} \) results of the test conducted with various clock frequencies. The \( C_1 \) constant scales linearly, while the \( C_2 \) constant responds to the trend line's slope.

\section{Conclusion}
\begin{itemize}
    \item Summary of key points
    \item Best practices for FPGA designers
\end{itemize}
Due to the exponential term \( e^{t_{\text{met}} / C_2} \), it has the largest effect on MTBF. The \( C_2 \) constant depends on the technology used to manufacture the device, with faster technologies generally having different characteristics. In most cases, metastability is not the main concern for designers.

To improve metastability MTBF, designers can increase \( t_{\text{met}} \) by adding extra register stages to the synchronizer chain. A three-stage synchronizer is typically recommended.

\newpage
\bibliographystyle{ieeetr}
\bibliography{references} % Add your references in a 'references.bib' file

\end{document}
