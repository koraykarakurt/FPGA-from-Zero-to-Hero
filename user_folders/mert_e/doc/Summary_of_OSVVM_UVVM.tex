\documentclass{article}
\usepackage{graphicx} % Required for inserting images

\title{OSVVM and UVVM Summary}
\author{Mert Ecevit}
\date{April 2025}

\begin{document}

\maketitle

\newpage

\section*{OSVVM Methodology}
\subsection{OSVVM Verification Framework}

Looks identical to a SystemVerilog framework:

\begin{itemize}
  \item \textbf{Test sequencer (\texttt{TestCtrl})} calls transactions = \textit{test case}
  \item Each test case is a separate \textbf{architecture} of \texttt{TestCtrl}
  \item \textbf{Verification components (VC)} implement \textit{interface signaling}
\end{itemize}

\begin{figure}
    \centering
    \includegraphics[width=1\linewidth]{Figures/verification_framework.png}
    \caption{OSVVM VERIFICATON FRAMEWORK}
    \label{fig:enter-label}
\end{figure}

\subsubsection{Elements}

\begin{itemize}
    \item Transaction Interface + Transaction API
    \item Verification Components
    \item Test Sequencer
\end{itemize}

\subsection{Transaction Interfaces and APIs}

Transaction interfaces written with VHDL record types and Transaction APIs written with VHDL procedures. 

\begin{itemize}
    \item The record is "inout" port
    \item Long term plan of for this part is to migrate VHDL-2019 Interfaces
\end{itemize}

Model Independent Transactions (MIT) define standardized interfaces and APIs that can be applied across multiple related protocols. 

For example, the \textbf{Address Bus MIT} is designed for memory-mapped interfaces such as AXI4 and Avalon, while the \textbf{Stream MIT} supports stream-based protocols like AXI Stream and UART. 

By using MIT-based Verification Components, the development of test case components becomes significantly more efficient and consistent.


\subsection{Test Sequencer = TestCtrl}

\begin{itemize}
  \item All test logic is kept in a single file for simplicity and maintainability.
  \item A control process governs the test execution flow, managing initialization and finalization.
  \item Each interface is assigned a separate process, enabling concurrency similar to the design.
  \item Each interface process is responsible for handling its own set of transactions.
  \item Test scenarios are constructed as sequences of transaction calls.
  \item The architecture supports the integration of:
  \begin{itemize}
    \item Directed tests,
    \item Constrained-random testing,
    \item Scoreboards for checking results,
    \item Functional coverage collection.
  \end{itemize}
  \item Synchronization mechanisms ensure proper coordination among parallel processes.
  \item Error reporting and messaging are included to assist in debugging and traceability.
\end{itemize}

\subsection{Constrained Random Testing}

\begin{itemize}
  \item Efficiently produces realistic stimuli suitable for driving the design under test.
  \item Well-suited for scenarios involving numerous similar or repeated structures.
  \item Capable of handling various types of data and operations such as:
  \begin{itemize}
    \item Operational modes,
    \item Instruction sequences,
    \item Network packet flows,
    \item Processor-level command execution.
  \end{itemize}
\end{itemize}

OSVVM has a Randomized Library to test and check. It is more easy to detect the errors.

\subsubsection{Scoreboards}
Self-checking when data is minimally transformed.

\begin{itemize}
  \item A built-in verification mechanism is used to ensure correct data transformations.
  \item This is typically implemented through customized FIFOs that include automatic checking logic.
  \item It is designed to be flexible by using package-level generics, allowing compatibility with various data types.
  \item The system supports advanced behaviors such as handling out-of-order data processing and managing dropped or lost values.
\end{itemize}

\subsection{Functional Coverage}

\begin{itemize}
  \item Cross coverage monitors how independent elements relate to one another.
  \item For example, it helps determine whether each register set has been tested with every ALU input.
  \item It involves writing code that checks whether key items in the test plan are being exercised.
  \item It ensures coverage of requirements, features, and edge conditions.
  \item Item coverage, also known as point coverage, focuses on variation within a single object.
  \item An example is categorizing transfer sizes into bins such as: 1, 2, 3, 4--127, 128--252, 253, 254, and 255.
  \item The purpose is to track what the randomized testing actually stimulated.
  \item A test is considered complete when both functional coverage and code coverage reach 100\%.
  \item Code coverage alone is not sufficient.
  \item Code coverage only reflects which lines of code were executed.
  \item It fails to capture conditions outside the code itself, such as bin coverage or uncorrelated combinations.
\end{itemize}

\subsection{Scripting System}

\begin{itemize}
  \item Designed to be compatible with various simulators, ensuring platform-independent usage.
  \item Provide automation for tasks such as library setup, code compilation, and simulation execution.
  \item Support a wide range of simulators including GHDL, NVC, Aldec, Siemens, Synopsys VCS, and Cadence Xcelium.
  \item Scripts are structured in a modular way to enhance reusability and maintainability.
\end{itemize}

\subsection{Reporting Features}

\begin{itemize}
    \item Mini-Report in Text Format
    \item Detailed HTML Report
    \begin{itemize}
        \item Overview of the complete build process
        \item Breakdown for each test suite
        \item Individual test case summaries within suites
    \end{itemize}
\end{itemize}

\subsubsection*{1. Build Summary Report (HTML Format)}
\begin{itemize}
    \item Comprehensive overview of the complete build execution
    \item Summarized results for all test suites and individual test cases
    \item Direct access points to in-depth test documentation
\end{itemize}

\subsubsection*{2. Detailed Test Case Documentation (HTML)}
\begin{itemize}
    \item System alert documentation highlighting errors and warnings
    \item Coverage analysis reports for functional verification
    \item Verification scoreboard results
    \item Quick-access links to simulation output transcripts
\end{itemize}

\subsubsection*{3. Enhanced Simulation Transcript (HTML)}
\begin{itemize}
    \item Critical errors prominently displayed in red
    \item Expandable/collapsible content sections
    \item Optimized for rapid analysis of extensive output logs
\end{itemize}

\subsubsection*{4. Requirements Verification Report}
\begin{itemize}
    \item Clear visualization of requirements coverage status
\end{itemize}

\section*{UVVM Methodology}
What is UVVM?
\begin{itemize}
    \item A comprehensive VHDL verification framework and methodology
    \item Freely available under open-source licensing
    \item Features well-organized system architecture and infrastructure
    \item Delivers substantial improvements in verification productivity
    \item Enhances final design quality and reliability
    \item Doulos-recommended approach for testbench development
    \item Currently being expanded through ESA (European Space Agency) initiatives
    \item IEEE Standards Association open-source development project
    \item Compatible with all VHDL-2008 compliant simulation tools
\end{itemize}

\begin{figure}
    \centering
    \includegraphics[width=1\linewidth]{Figures/uvvm_table.png}
    \caption{UVVM System}
    \label{fig:enter-label}
\end{figure}

\subsection{Utilities}
\begin{itemize}
    \item \texttt{check\_stable()}
    \item \texttt{await\_stable()}
    \item \texttt{clock\_generator()}
    \item \texttt{adjustable\_clock\_generator()}
    \item \texttt{random()}
    \item \texttt{randomize()}
    \item \texttt{gen\_pulse()}
    \item \texttt{block\_flag()}
    \item \texttt{unblock\_flag()}
    \item \texttt{await\_unblock\_flag()}
    \item \texttt{await\_barrier()}
    \item \texttt{enable\_log\_msg()}
    \item \texttt{disable\_log\_msg()}
    \item \texttt{to\_string()}
    \item \texttt{fill\_string()}
    \item \texttt{to\_upper()}
    \item \texttt{replace()}
    \item \texttt{normalize\_and\_check()}
    \item \texttt{set\_log\_file\_name()}
    \item \texttt{set\_alert\_file\_name()}
    \item \texttt{wait\_until\_given\_time\_after\_rising\_edge()}
\end{itemize}

\begin{figure}
    \centering
    \includegraphics[width=1\linewidth]{Figures/data_comm.png}
    \caption{Simple Data Communication}
    \label{fig:enter-label}
\end{figure}

\subsection{Bus Functional Models(BFMs)}
\begin{itemize}
    \item \textbf{Advantages for Simple Testbenches:}
    \begin{itemize}
        \item Implemented as specialized procedures in dedicated packages
        \item Straightforward invocation from any process
    \end{itemize}
    
    \item \textbf{Limitations:}
    \begin{itemize}
        \item Processes have sequential execution constraints
        \item Can only perform one BFM operation at a time
        \item Cannot simultaneously execute multiple BFMs or other operations
    \end{itemize}
    
    \item \textbf{Solution for Concurrent Operations:}
    \begin{itemize}
        \item Requires implementation as separate entities/components
        \item Verification Components (VCs) provide this capability
    \end{itemize}
\end{itemize}

\subsection{VHDL Verification Components(VVCs)}
VVCs enhance traditional BFMs with advanced capabilities through a structured design:

\subsubsection*{Core Components}
\begin{itemize}
    \item \textbf{Command Interpreter:} Analyzes and decodes incoming commands
    \item \textbf{Command Queue:} Manages pending operations in FIFO order  
    \item \textbf{Command Executor:} Processes commands using the underlying BFM
\end{itemize}

\subsubsection*{Advanced Features}
\begin{itemize}
    \item Configurable delays between commands for timing verification
    \item Queue management for handling multiple concurrent operations
    \item Automatic detection of command sequence completion
    \item Real-time interface activity monitoring
    \item Multicast/broadcast support for multi-agent control
    \item Detailed transaction logging and reporting
    \item Built-in sequencers for protocol automation
\end{itemize}

\subsubsection*{Extension Capabilities}
\begin{itemize}
    \item Plug-in protocol checkers (timing, framing, intervals)
    \item Custom sequence generator integration
    \item Support for complex transactions:
    \begin{itemize}
        \item Split transactions
        \item Out-of-order execution
    \end{itemize}
\end{itemize}

\subsection{Advanced Functions of UVVM}

\begin{itemize}
    \item \textbf{Scoreboards} - Verification data checking
    \item \textbf{Monitors} - Interface activity observation
    \item \textbf{Randomization Control} - Functional coverage management
    \item \textbf{Error Injection}:
    \begin{itemize}
        \item Brute force methods
        \item Protocol-aware techniques
    \end{itemize}
    \item \textbf{Local Sequencers} - Command sequence automation
    \item \textbf{Property Checkers} - Protocol rule verification
    \item \textbf{Transaction Info} - Detailed operation logging
    \item \textbf{Watchdogs}:
    \begin{itemize}
        \item Simple timeout detection
        \item Activity-based monitoring
    \end{itemize}
    \item \textbf{Hierarchical VVCs} - With integrated scoreboards
    \item \textbf{Specification Coverage} - Requirements-to-test tracing
\end{itemize}

\subsection{Specification Coverage}

\subsubsection*{Verification Steps}
\begin{enumerate}
    \item Specify all requirements
    \item Report coverage from test sequencers (or other testbench components)
    \item Generate summary reports
\end{enumerate}

\subsubsection*{Coverage Reporting}
\begin{itemize}
    \item Coverage metrics per individual requirement
    \item Test cases associated with each requirement
    \item Requirements verified by each test case
    \item Cumulative coverage across multiple test cases
\end{itemize}

\subsection{2nd ESA Project}
\begin{itemize}
    \item \textbf{Advanced randomization} implemented with simple interface
    
    \begin{itemize}
        \item \textbf{Optimized Randomization}:
        \begin{itemize}
            \item Randomization without replacement
            \item Weighted according to:
            \begin{itemize}
                \item Target distribution
                \item Previous events
            \end{itemize}
            \item Achieves the lowest number of randomizations needed for a given target
        \end{itemize}
        
        \item \textbf{Functional Coverage}:
        \begin{itemize}
            \item Verifies that all specified scenarios have been tested
        \end{itemize}
    \end{itemize}
\end{itemize}

Also the Enhanced Randomisation of UVVM is structed.

\begin{itemize}
    \item \textbf{Seamless integration} with UVVM framework
    \begin{itemize}
        \item Unified \textbf{alert handling} system
        \item Consolidated \textbf{logging} mechanism
    \end{itemize}
    
    \item \textbf{Enhanced clarity and usability}
    \begin{itemize}
        \item Intuitive \textbf{keyword system} for improved comprehension
        \item Clean, organized output for better \textbf{overview}
    \end{itemize}
    
    \item \textbf{Maintainability advantages}
    \begin{itemize}
        \item Modular design for \textbf{easy extension}
        \item Straightforward \textbf{code maintenance}
    \end{itemize}
\end{itemize}

\end{document}
